
\chapter{Implementation}

\section{Code Instrumentation}

The problem arises, because the management code uses the same libraries as the application. When for example RECORDER writes an event to a file, this should not trigger additional events.  
\subsection{Routines}
\subsection{Attribute Accesses}
\subsection{Attribute Manipulation from C Level}
\section{Architecture}
%unterschiedliche Parts noch genauer beschreiben
%-Log: (mikro-architektur), wieso kein XML?
\section{Advantages over Traditional Capture and Replay}
\section{Limitations}
%unter welchen Umstaenden musste manuell instrumentiert werden??
\section{Future Steps}
%Fehlende features + idee zu deren implementierung

\section{Building the Example from Source}
In this section the process of building an example with capture replay support under Linux will be described. Setting up the modified version of Eiffel Studio and all necessary tools will take the biggest part of this explanation.\\
%TODO: wieso genau wird eine modifizierte Version von Eiffel Studio benoetigt??
It is assumed that all commands in the following steps are executed in the same session, thus keeping the environment variables.\\

\subsection{Building the Preliminaries}
The first tools we need are the  Eiffel Studio Tools \cite{estudiotools}. These will be used in many setup scripts in the examples or tests from the repository. Install those tools according to the description on the webpage.\\

The delivery of the modified Eiffel Studio was built using revision 69201 of Eiffel Studio. Building a delivery with a later version of Eiffel Studio was not testet, so it might not work. If no binaries of revision 69201 are available, a delivery of that revision from source \texttt{(stimmt das wirklich??? gaebe das kein Henne / Ei Problem?)}.\\
After copying the delivery to ~/estudio/Eiffel60\_gpl\_69201, it can be activated:
\bashlisting
\begin{lstlisting}
   activate_estudio 60_gpl_69201
\end{lstlisting}

Now Gobo \cite{gobo} can be installed can be installed from svn (Revision 6001 was successfully tested).
- Install Gobo  from svn (revision 6001) -->
\begin{lstlisting}
svn co -r6001 https://gobo-eiffel.svn.sourceforge.net/svnroot/gobo-eiffel/gobo/trunk ~/capture_replay/gobo
export GOBO=~/capture\_replay/gobo
export PATH=$GOBO/bin:$PATH
$GOBO/work/bootstrap/bootstrap.sh gcc ise
\end{lstlisting}

As all preliminaries are installed, Erl-G \cite{erlg} can be downloaded and built. Revision 719 of Erl-G was tested together with capture and replay.
\begin{lstlisting}
svn co -r719 https://svn.origo.ethz.ch/autotest/trunk/erl_g ~/capture_replay/erl_g
export ERL_G=~/capture_replay/erl_g
export PATH=$ERL_G/bin:$PATH
cd $ERL_G
#EIFFEL_SRC is needed. avoid conflicts between EIFFEL_SRC and ISE_LIBRARY.
export ISE_LIBRARY=$EIFFEL_SRC
geant install
geant compile
\title{Selective Capture and Replay for Eiffel
\end{lstlisting}

To build a delivery of the modified Eiffel Studio, execute these commands: (this will take a few hours).
\begin{lstlisting}
cd ~/capture_replay/
mkdir es
svn co https://eiffelsoftware.origo.ethz.ch/svn/es/branches/capture_replay es
export EIFFEL_SRC=~/capture_replay/es/Src
cd es
geant -b $EIFFEL_SRC/scripts/build.eant build_es
\end{lstlisting}

Before an example can be built the delivery that was just created needs to used be set as default instance of Eiffel Studio
\begin{lstlisting}
cd ~/estudio
ln -s ~/capture_replay/es/EiffelXX EiffelCR
activate_estudio CR
\end{lstlisting}

In order to make the created Eiffel Studio use a modified version of the runtime, it is necessary to recompile the runtime with modified CFLAGS. The new version of the runtime then needs to be installed in the delivery.

It is not possible to directly build Eiffel Studio with the modified version of the runtime, because the changes in the runtime are not compatible to Eiffels store mechanism (\texttt{TODO: das auch noch unter irgendwelchem future work anmerken...}). This would render Eiffel Studio unusable because it relies on this mechanism during the build process.
\begin{lstlisting}
export CFLAGS='-DCAPTURE_REPLAY' 
cd $EIFFEL_SRC
#build the runtime from scratch (clobber the old one)
geant -b scripts/build.eant compile_runtime
cd $ISE_EIFFEL/studio/spec/linux-x86/lib
rm *
cp $EIFFEL_SRC/C/run-time/lib* .
\end{lstlisting}


\subsection{Building an Example}
Now, all necessary tools should be installed and the corresponding environment variables set. And we can start to build an example.\\
First we need to add reflection support to the example project. Erl G will generate reflection classes for us. If the environment variables are correctly set, the geant script should invoke Erl G automatically. \\
At the moment Erl-G does not support overrides because this feature is missing in the Gobo parser. Therefore it's necessary to override the necessary classes manually. There are two geant tasks that take care of this:

\begin{itemize}
\item \identifier{patch\_elks} makes the manual override by copying the modified elks classes from \$EIFFEL\_SRC/library/base/capture\_replay/elks\_overrides to \$ISE\_LIBRARY/library/base/elks \\
\item \identifier{unpatch\_elks} restores the original state by copying the original elks classes from \$EIFFEL\_SRC/library/base/elks to \$ISE\_LIBRARY/library/base/elks .
\end{itemize}

\begin{lstlisting}
cd ~/capture_replay/es/examples/capture_replay/iteration1
geant install
\end{lstlisting}

The example can now be opened with the modified version of Eiffel Studio. Make sure that the CFLAGS are still set to '-DCAPTURE\_REPLAY'. Otherwise it will not be possible to build the example.



\section {Limitations}
%-Konstruktoren nach ANY exportiert (fehlende unterstuetzung von Eiffel-Seite fuer Konstruktoren)
%- Access modifiers e.g. export of a observed feature only to an unobserved class --> replay not possible.