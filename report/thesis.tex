\documentclass[a4paper,10pt]{report}

%picture imports
\usepackage{graphicx}
\usepackage{color}
\definecolor{lbcolor}{rgb}{0.85,0.85,0.85}

%sourcecode formatting
\usepackage[final]{listings}
\usepackage{url}


\lstloadlanguages{Eiffel,bash}
\newcommand{\eiffellisting}{\lstset{language=Eiffel,tabsize=2,backgroundcolor=\color{lbcolor},frame=tb,breaklines=true}}
\newcommand{\bashlisting}{\lstset{basicstyle=\small\tt,language=bash,tabsize=2,backgroundcolor=\color{lbcolor},frame=tb,breaklines=true,morekeywords={svn,geant,mkdir,activate_estudio}}}
\newcommand{\identifier}[1]{\texttt{#1}}

\eiffellisting
\begin{document}

%opening
\title{Selective Capture and Replay for Eiffel}
\author{Stefan Sieber}

\begin{titlepage}
	\begin{flushleft}
		\includegraphics[width=100pt]{illustrations/SE.png}
	\end{flushleft}

	\vskip 50pt
	\begin{center}
		\Huge \textbf{Capture and Replay for Eiffel}
	\end{center}

	\vskip 100pt
	\begin{flushright}
		\large \textbf{Stefan Sieber}\\
		\vskip 10pt
		02-911-790\\
		Master Thesis\\
		Summer 2007\\
	\end{flushright}
	
	\vskip 20pt
	\begin{flushright}
		\large Chair of Software Engineering\\
		Department of Computer Science\\
		ETH Z\"{u}rich
	\end{flushright}
		
	\vskip 20pt
	\begin{flushright}
		\large Andreas Leitner\\
		Prof. Bertrand Meyer
	\end{flushright}

	\vskip 50pt
	\begin{tabular*}{1.00\textwidth}{@{\extracolsep{\fill}}lr}
		 \includegraphics[width=100pt]{illustrations/ETH.png} &
		 \includegraphics[width=100pt]{illustrations/INF.png}
	\end{tabular*}
	\vfil
\end{titlepage}

\newpage
\begin{abstract}
%Should the motivation include a reference to cdd?
Debugging applications is very tedious work. One of the hardest steps to find a fault in a program is to reproduce the steps that lead to the associated program failure. Capture and replay aims at simplifying this process. When trying to replay a software it is necessary to decide what parts are to be simulated in order to drive the rest of the program. 

Selective Capture and Replay \cite{orso05may} makes it possible to replay the same program run again on a predefined set of classes, thus reproducing those steps automatically. It is possible to define the subset of the program that should be replayed individually. This makes it special compared to conventional systems where this border is fixed (e.g. everything except the network library). Another unique feature is that the amount of recorded data is linear to the number of variables passed over this border, whereas conventional systems record all the data that passes the border.
 The goal of this thesis is to implement Selective Capture Replay for Eiffel, which in contrast to Java, supports multiple inheritance. This makes it necessary to instrument the programs in a different way than in the original implementation. One benefit of this changed implementation is the possibility to switch between capture and replay phase without recompilation.
 To validate the implementation, an example will be presented and the performance of the technique will be measured using this example.

%CDD [reference!] facilitates this step by providing the developer a testcase that results in the same failure. 
%motivation
%capture/replay
%scope
%-modifications
%-implementation
%-experimental results
\end{abstract}

\newpage

\section{Introduction}

\chapter{Related Work}
\section{Overview}
Because replaying application runs is important in order to be able to debug or test applications, there exist many techniques that implement capture and replay. As mentioned before, one of the basic steps in order to be able to capture and replay an application is to distinct between the deterministic core (the \emph{observed part}) and the non-deterministic environment (the \emph{unobserved part}) of the application like user input, network or external storage.

In general, capture and replay can be divided into two phases: The \emph{capture phase}, where the application is run and the information that is needed for replaying the application is captured and the \emph{replay phase} where the application is replayed based on this information.

During \emph{capture phase}, the capture and replay implementation needs to record the information that will later be needed to replay the observed part. In general this is at least all information that is passed from the unobserved part to the observed part. The information is captured by some management code that was introduced by the capture and replay framework (\figref{fig:GenericCrStructure_capture}).
\begin{figure}[h]
  \centering
  \includegraphics[width=0.5\textwidth]{illustrations/capture_and_replay_generic_structure_capture}
  \caption{Generic structure of a capture and replay technique - capture phase}
  \label{fig:GenericCrStructure_capture}
%\includegraphics{illustrations/capture_and_replay_generic_structure}
\end{figure}

During \emph{replay phase} the management code needs to replace the unobserved part ( 
\figref{fig:GenericCrStructure_replay}) in order to replay the run of the observed part. Depending on the part of the application that was defined to be unobserved, the management part can act both as a driver (e.g. in the case of mouse events) and stub (e.g. in the case of a network socket). 
%Schema dazu: Capture phase & Replay phase , observed & unobserved part && log
\begin{figure}[h]
  \centering
  \includegraphics[width=0.4\textwidth]{illustrations/capture_and_replay_generic_structure_replay}
  \caption{Generic structure of a capture and replay technique - replay phase}
  \label{fig:GenericCrStructure_replay}
%\includegraphics{illustrations/capture_and_replay_generic_structure}
\end{figure}

Different implementations make different assumptions about what behaves the same (is deterministic) and what changes its behaviour throughout different runs of the application. Therefore they define different portions of the program as observed and unobserved part. Here we will categorize the different implementation based on what is defined to be unobserved.

\section {Capturing User Input}
The best known capture and replay technique is used for GUI testing. Here, keyboard and mouse events are considered to belong to the unobserved part. The exact location where these events are captured may vary (inside the application or through the operating system), but the advantages and limitations mostly stay the same. Usually, capture and replay of user input is used for regression testing. In that setup, the tester executes a sequence of actions on the applications GUI while capturing is enabled. These recorded actions can be replayed on the application in order to check if there were regressions. Abbot \cite{abbot} is an example of such a tool, although it offers more features than only capture and replay of user interactions.
\subsection{Advantages}
This technique is easy to understand, because it is based on a simple abstraction.
\subsection{Limitations}
If this technique is implemented in a very naive way (i.e. only capturing the location of mouse events, not the targets), a changed GUI layout renders a recorded run unusable. This is especially painful, if the capture and replay is used for regression tests. Another problem is that most applications use more than just the GUI to interact with their environment. As soon as the application uses the file system or the network, too, it is necessary to make sure, that the behaviour of this environment is the same every time the program is replayed. Otherwise the assumption, that the observed space behaves deterministically would be broken, which results in an incorrect replay of the application.

\section {Capturing Interactions with the Libraries}
JRapture \cite{jrapture} uses an approach that has the potential of replaying more complex applications. JRapture is a tool for capturing and replaying Java applications in the field. In addition to capturing and replaying, it also offers a profiling interface that permits the program to be instrumented for profiling in the replay phase.

In our model of capture and replay techniques, JRapture draws the border between observed and unobserved part between the Java API and the core of the program. They achieve this by instrumenting the Java API classes. JRapture supports multithreaded applications, but it does not guarantee a deterministic replay of concurrent applications.

\subsection{Advantages}
JRapture has the potential to replay more complex applications that involve file and network access. In principle it is able to replay every interaction of the observed part with its environment as long as these interactions are executed through the Java API.
\subsection{Limitations}
JRapture relies on a modifed version of the Java API, which was created by manual instrumentation of it. Only a subset of the Java API is covered by this instrumentation yet, for example it lacks support for network I/O at the moment. Thus although JRaptures approach has the potential of replaying complex applications, it does not succeed yet to do so. Another drawback of the manual instrumentation is that applications that interact with the environment through other mechanisms than the Java API (for example through the Java Native Interface JNI) can not be supported without additional manual instrumentations.
%\section {Capturing Interactions with the Operating System}
%Operating System - Library
%TODO: gehoert bugnet hier auch rein???
%state (wird das als cr bezeichnet?) vs. events
\section {Capturing Interactions with the Scheduler}
The techniques presented so far did not consider thread scheduling as another source of non-determinism. DejaVu \cite{dejavu}, a capture and replay tool for Java, considers thread scheduling as its only source of non-determinism, thus thread scheduling belongs to its unobserved part. Because threads are often scheduled by the operating system, it is not easy to instrument the scheduler in order to detect thread switches. DejaVu therefore introduces the concept of \emph{logical thread schedule} which is a simplified version of the real thread schedule (the \emph{physical thread schedule}). The \emph{logical thread schedule} contains enough thread schedule information to reproduce the execution behaviour of the program under the assumption that the thread schedule is the only source of non-determinism. By detecting some critical events during capture phase such as access to shared variables, and synchronization events, DejaVu is able to deduce this logical thread schedule.

%XXX Rettet der 2. Satz unser Modell????
DejaVu does not completely fit into our generic schema of a capture and replay technique from the implementation point of view, because the information passed between unobserved part (scheduler) and observed part (the application) is not directly captured but a simplified version is deduced. However, the semantic stays the same because the part of information that matters to the application is captured and can be used to replay the program afterwards.

\subsection{Advantages}
DejaVu succeeds in capturing and replaying a concurrent application that does not use any other source of non-determinism than the thread scheduler. Of the techniques listed here, DejaVu is the only one that takes thread scheduling into account.
\subsection{Limitations}
Because DejaVu only focuses on thread scheduling as a source of non-determinism of a program, it is not suited for capturing and replaying a general program. Nevertheless the technique of DejaVu could be combined with another capture and replay technique in order to build a system that allows deterministic replays of general multithreaded programs.

\section{Selective Capture Replay}
The capture and replay techniques seen so far all define a border between observed and unobserved part. In contrast to these techniques, \emph{Selective Capture and Replay} \cite{orso05may} implemented for java, offers the user the possibility to make its own definition of observed and unobserved part of the system. Observed and unobserved part are both defined as a set of classes. The interactions between observed and unobserved part are captured using automated code instrumentation.
\subsection{Advantages}
The main advantage of this technique is its flexibility. The possibility to individually define observed and unobserved part makes it possible to minimize the amount of data that is exchanged between these two parts by a smart choice of the border. Because no custom capture mechanism is needed, applications that use own libraries to interact with the non-deterministic environment can be supported without manual instrumentation. 
\subsection{Limitations}
Although Selective Capture and Replay is a very generic approach, it assumes that multithreading does not cause non-deterministic behaviour of the observed part. Thus it is not generally possible to replay a program run that involves data races.
%Wir beachten call durations nicht (!)

%Überblick über das Paper
%Limitations (polymorphism / dynamic binding) --> oder doch nicht? she. Annahme über observed set.
%Benefit for cdd

%Überblick über das Paper
%Limitations (polymorphism / dynamic binding) --> oder doch nicht? she. Annahme über observed set.
%Benefit for cdd


\section{Capture and Replay for Eiffel}
\subsection{Relevant Differences between Eiffel and Java}
\subsection{Differences in the Requirements}
\subsection{Idea for Code Instrumentation}

% Unterschiedliche Language features Eiffel/Java
% Unterschiede der Anforderungen
% -keine dynamische Kompilierung
%  capture/replay ohne neukompilierung (cdd)

\chapter{Implementation}

\section{Code Instrumentation}

The problem arises, because the management code uses the same libraries as the application. When for example RECORDER writes an event to a file, this should not trigger additional events.  
\subsection{Routines}
\subsection{Attribute Accesses}
\subsection{Attribute Manipulation from C Level}
\section{Architecture}
%unterschiedliche Parts noch genauer beschreiben
%-Log: (mikro-architektur), wieso kein XML?
\section{Advantages over Traditional Capture and Replay}
\section{Limitations}
%unter welchen Umstaenden musste manuell instrumentiert werden??
\section{Future Steps}
%Fehlende features + idee zu deren implementierung

\section{Building the Example from Source}
In this section the process of building an example with capture replay support under Linux will be described. Setting up the modified version of Eiffel Studio and all necessary tools will take the biggest part of this explanation.\\
%TODO: wieso genau wird eine modifizierte Version von Eiffel Studio benoetigt??
It is assumed that all commands in the following steps are executed in the same session, thus keeping the environment variables.\\

\subsection{Building the Preliminaries}
The first tools we need are the  Eiffel Studio Tools \cite{estudiotools}. These will be used in many setup scripts in the examples or tests from the repository. Install those tools according to the description on the webpage.\\

The delivery of the modified Eiffel Studio was built using revision 69201 of Eiffel Studio. Building a delivery with a later version of Eiffel Studio was not testet, so it might not work. If no binaries of revision 69201 are available, a delivery of that revision from source \texttt{(stimmt das wirklich??? gaebe das kein Henne / Ei Problem?)}.\\
After copying the delivery to ~/estudio/Eiffel60\_gpl\_69201, it can be activated:
\bashlisting
\begin{lstlisting}
   activate_estudio 60_gpl_69201
\end{lstlisting}

Now Gobo \cite{gobo} can be installed can be installed from svn (Revision 6001 was successfully tested).
- Install Gobo  from svn (revision 6001) -->
\begin{lstlisting}
svn co -r6001 https://gobo-eiffel.svn.sourceforge.net/svnroot/gobo-eiffel/gobo/trunk ~/capture_replay/gobo
export GOBO=~/capture\_replay/gobo
export PATH=$GOBO/bin:$PATH
$GOBO/work/bootstrap/bootstrap.sh gcc ise
\end{lstlisting}

As all preliminaries are installed, Erl-G \cite{erlg} can be downloaded and built. Revision 719 of Erl-G was tested together with capture and replay.
\begin{lstlisting}
svn co -r719 https://svn.origo.ethz.ch/autotest/trunk/erl_g ~/capture_replay/erl_g
export ERL_G=~/capture_replay/erl_g
export PATH=$ERL_G/bin:$PATH
cd $ERL_G
#EIFFEL_SRC is needed. avoid conflicts between EIFFEL_SRC and ISE_LIBRARY.
export ISE_LIBRARY=$EIFFEL_SRC
geant install
geant compile
\title{Selective Capture and Replay for Eiffel
\end{lstlisting}

To build a delivery of the modified Eiffel Studio, execute these commands: (this will take a few hours).
\begin{lstlisting}
cd ~/capture_replay/
mkdir es
svn co https://eiffelsoftware.origo.ethz.ch/svn/es/branches/capture_replay es
export EIFFEL_SRC=~/capture_replay/es/Src
cd es
geant -b $EIFFEL_SRC/scripts/build.eant build_es
\end{lstlisting}

Before an example can be built the delivery that was just created needs to used be set as default instance of Eiffel Studio
\begin{lstlisting}
cd ~/estudio
ln -s ~/capture_replay/es/EiffelXX EiffelCR
activate_estudio CR
\end{lstlisting}

In order to make the created Eiffel Studio use a modified version of the runtime, it is necessary to recompile the runtime with modified CFLAGS. The new version of the runtime then needs to be installed in the delivery.

It is not possible to directly build Eiffel Studio with the modified version of the runtime, because the changes in the runtime are not compatible to Eiffels store mechanism (\texttt{TODO: das auch noch unter irgendwelchem future work anmerken...}). This would render Eiffel Studio unusable because it relies on this mechanism during the build process.
\begin{lstlisting}
export CFLAGS='-DCAPTURE_REPLAY' 
cd $EIFFEL_SRC
#build the runtime from scratch (clobber the old one)
geant -b scripts/build.eant compile_runtime
cd $ISE_EIFFEL/studio/spec/linux-x86/lib
rm *
cp $EIFFEL_SRC/C/run-time/lib* .
\end{lstlisting}


\subsection{Building an Example}
Now, all necessary tools should be installed and the corresponding environment variables set. And we can start to build an example.\\
First we need to add reflection support to the example project. Erl G will generate reflection classes for us. If the environment variables are correctly set, the geant script should invoke Erl G automatically. \\
At the moment Erl-G does not support overrides because this feature is missing in the Gobo parser. Therefore it's necessary to override the necessary classes manually. There are two geant tasks that take care of this:

\begin{itemize}
\item \identifier{patch\_elks} makes the manual override by copying the modified elks classes from \$EIFFEL\_SRC/library/base/capture\_replay/elks\_overrides to \$ISE\_LIBRARY/library/base/elks \\
\item \identifier{unpatch\_elks} restores the original state by copying the original elks classes from \$EIFFEL\_SRC/library/base/elks to \$ISE\_LIBRARY/library/base/elks .
\end{itemize}

\begin{lstlisting}
cd ~/capture_replay/es/examples/capture_replay/iteration1
geant install
\end{lstlisting}

The example can now be opened with the modified version of Eiffel Studio. Make sure that the CFLAGS are still set to '-DCAPTURE\_REPLAY'. Otherwise it will not be possible to build the example.



\section {Limitations}
%-Konstruktoren nach ANY exportiert (fehlende unterstuetzung von Eiffel-Seite fuer Konstruktoren)
%- Access modifiers e.g. export of a observed feature only to an unobserved class --> replay not possible.

\section{Experimental Results}
\subsection{Example Application}
\subsection{Results}
\subsection{Conclusion}
%Eiffel Vision Beispiel, mit unterschiedlichen observed sets

\section{Future Work}
%Fehlende C/R und Language Features beschreiben, sowie einen möglichen Lösungsansatz.
\section{Appendix}

\subsection{Log File Grammar}

log ::= event* \\
event ::= call $\mid$ return $\mid$ outread\\
call ::= calltype entity methodname entity* \texttt{“\%N”} \\
return ::= returntype [entity]  \texttt{“\%N”}\\
outread ::= \texttt{OUTREAD} entity attribute\_name entity
calltype ::= \texttt{INCALL} $\mid$ \texttt{OUTCALL} \\
returntype ::= \texttt{INCALLRET} $\mid$ \texttt{OUTCALLRET} \\
methodname ::= identifier \\
attribute\_name ::= identifier \\
entity ::= (object $\mid$ value) \\
size ::= integer \\
object ::= \texttt{[NON$\_$BASIC} typename object\_id \texttt{]} \\
value ::= \texttt{[BASIC} typename \texttt{"} string \texttt{"]} \\
typename ::= identifier \\
object\_id ::= integer\\
identifier ::= [A-Za-z]character*\\
string ::= character*\\

\subsection{capture-phase performance measurements}
\begin{itemize}
	\item normal application: 2.5s
	\item Captured application: 30s
	\begin{itemize}
		\item RECORDER: 1.7s\\
		\item SERIALIZER: 26s
		\begin{itemize}
			\item write$\_$statements: 16s
			\begin{itemize}
				\item file.put$\_$string: 4s
				\item object$\_$id: 7s
				\item other: 5s
			\end{itemize}
			\item is$\_$basic$\_$type: 3s
			\item other: 7s
		\end{itemize}
	\end{itemize}
\end{itemize}

\bibliographystyle{plain} 
\bibliography{thesis}

\end{document}


